\chapter{Project management}

In order to successfully actualize this project, it is necessary to establish a common, high-level understanding of the team, external factors influencing the project, the project itself, and how to organize the work. The project management chapter is meant to document this.

\section{Team Structure}
We are a very small development team of just four team members. In a typical software development project, there are a large number of different roles, so each of us have to be assigned several different roles. We have defined a role for each responsibility that we believe to be important in our process. The person who is assigned a role should have an overview of the progress and have control over which tasks need to be done on their area. The work itself may be broken down into tasks so that these tasks can be executed by anyone from the team. The roles in our team are listed in the table \ref{table-teamroles}, overview of the roles is in the table \ref{table-rolesoverview}.

\begin{table}
\centering
\begin{tabular}{ l  l }
  \hline
  \textbf{Team member} & \textbf{Roles} \\
  Ivo & Group leader, Customer and Advisor Contact, Scrum Master \\
  Oystein & Test Manager, QA Manager, Weekly Report Manager \\
  Oddvar & GUI Designer, Code Master, Meeting Secretary, Time Keeper \\
  Martin & System Architect, Report Manager \\
  \hline
\end{tabular}
\label{table-rolesoverview}
\caption{Team role overview}
\end{table}


\begin{table}
\begin{tabularx}{\textwidth}{ | l | X | l | }
  \hline
  \textbf{Role} & \textbf{Description} & \textbf{Assignee} \\ 
  \hline
  Team leader & Is responsible for administrative tasks and makes the final decisions. & Ivo \\ 
  \hline
  Scrum Master & Shields the development team from external distractions and enforces the Scrum scheme.  & Ivo \\ 
  \hline
  Customer Contact & Handles communication with the customer. The customer should contact this person regarding general requests, questions and reminders. & Ivo (backup Martin) \\ 
  \hline
  Advisor Contact & Handles communication with the advisor. The advisor should contact this person regarding general requests, questions and reminders.  & Ivo (backup Martin) \\ 
  \hline
  System Architect & Is responsible for the system architecture including distinctions and relations between subsystems and general code design choices. & Martin \\ 
  \hline
  Code Master & Overall responsible for code management and structure. Managing branches in Git repository. & Oddvar  \\ 
  \hline
  GUI Designer & Is responsible for the layout and design of graphical user interfaces. & Oddvar \\ 
  \hline
  Test Manager & Is responsible for testing including unit tests, integration tests and usability tests. & Øystein \\ 
  \hline
  Report Manager & Is responsible for delegating and overseeing work on the project report. & Martin \\ 
  \hline
  Customer Representative & Participates in regular meetings to discuss the progress, project status and future tasks. Represents the customer. & Peder Kongelf \\ 
  \hline
  Customer Technical Advisor & May be consulted about technical aspects of the project. & Stig Lau \\ 
  \hline
  Advisor & Serves as a one-man steering committee for the project. & Meng Zhu \\ 
  \hline
  Meeting Secretary & Is responsible for making sure notes get written and sent after each meeting with the advisor and customer. & Oddvar \\ 
  \hline
  Quality Assurance Manager &  & Øystein \\ 
  \hline
  Weekly Report Writer & Is responsible for finalizing the weekly report(s) for the advisor and customer, and getting these delivered for approval. Also responsible for meeting agendas and their delivery. & Øystein \\ 
  \hline
  Time Keeper & Responsible for making sure that everybody is logging their work, and logging team activities. & Oddvar \\ 
  \hline
\end{tabularx}
\caption{Team role}
\label{table-teamroles}
\end{table}

\section{Concrete Project Work Plan}
The first 4,5 weeks of the project were used on project planning, pre- study on the domain of the problem, to deduce user stories for the product backlog and to create the overall architecture of the system. The project is a proof of concept type of project, so it was important that we used a considerable amount of time on studying the domain of the problem and to find technologies that would solve the task as best as possible. Our Scrum process consists of 4 sprints, each 2 weeks long. In each sprint we had 200 work hours available, but a part of this will be used on report writing, project management, lectures and meetings. The two first sprint had an estimate of 130 hours available for development tasks, while the two last had an estimate of 140 work hours available. The last week will be used to evaluate the project, finish the report and to prepare the final presentation. The timeline of the project is outlined in Figure~\ref{figure:gantt}, and the WBS of the project is given in Table~\ref{table:wbs}.

\section{Schedule of Results}
\subsection {Deliverables}
These are the deliverables and deadlines, that we have to take into account.
\begin {itemize}

\item August 21, Project start

\item October 14, Pre- Delivery: Deliver a copy of the Abstract, Introduction, the Pre-study and the Choice-of Lifecycle-model chapters to the external examiner (censor) and technical writing teacher. Also deliver the outline of the full report (Table of  Contents).

\item November 22, Final Delivery: Project end. Deliver final report and present and demonstrate the final product at NTNU. Four printed and bound copies of  the project report should be delivered, as well as one electronic (PDF) copy.

\end {itemize}

\subsection {Sprints}

Sprint deadlines:
The pre- study, requirements, and testing plan activities should be finished before the start of the first sprint. If this is not the case the number sprints and their deadline might change. The start and end dates of each sprint is listed in Table~\ref{table:sprintdeadlines}

\begin{table}
\caption{Sprint Deadlines}
\centering
\begin{tabular}{ l l l }
\hline
Sprint Nr.		&Start		&End		\\
\hline
1		&24. September		&5. October		\\
2		&8. September			&19. October		\\
3		&22. October			&2. November		\\
4		&4. November			&18. November	\\
\hline
\end{tabular}
\label{table:sprintdeadlines}
\end{table}

\begin{figure}
\centering
\includegraphics[width=6in]{image/gantt.png}
\caption{Gantt Chart}
\label{figure:gantt}
\end{figure}

\begin{table}
\caption{Work Breakdown Structure}
\centering
\begin{tabular}{ l l l l l }
\hline 
			&				&				&\multicolumn{2}{c}{Hours}		\\
 Task		& From date		&To date			&Est.			&Act.	                \\ 
\hline \\ [-2.0ex]
 The report     			&21/08/2012		&22/11/2012		&300		&         	 \\
 Project Management	&21/08/2012		&22/11/2012		&180		&		\\
 Project Planning		&21/08/2012		&05/09/2012		&100		&		\\
 Lectures				&21/08/2012		&05/09/2012		&40			&		\\	
 Pre- Study			&06/09/2012		&18/09/2012		&80			&		\\
 Backlog				&10/09/2012		&21/09/2012		&60			&		\\
 Architecture			&17/09/2012		&21/09/2012		&40			&		\\
\hline \\ [-2.0ex]
 \bf{Sprint 1}			&\bf{24/09/2012}	&\bf{05/10/2012}	&\bf{130}		&		\\
 Planning				&				&				&20			&		\\
 Design/Implementation	&				&				&90			&		\\
 Testing				&				&				&20			&		\\
\hline \\ [-2.0ex]
 \bf{Sprint 2}			&\bf{08/10/2012}	&\bf{19/10/2012}	&\bf{130}		&		\\
 Planning				&				&				&20			&		\\
 Design/Implementation	&				&				&90			&		\\
 Testing				&				&				&20			&		\\
\hline \\ [-2.0ex]
 \bf{Sprint 3}			&\bf{22/10/2012}	&\bf{02/11/2012}	&\bf{140}		&		\\
 Planning				&				&				&20			&		\\
 Design/Implementation	&				&				&100		&		\\
 Testing				&				&				&20			&		\\
\hline \\ [-2.0ex]
 \bf{Sprint 4}			&\bf{05/11/2012}	&\bf{16/11/2012}	&\bf{140}		&		\\
 Planning				&				&				&20			&		\\
 Design/Implementation	&				&				&100		&		\\
 Testing				&				&				&20			&		\\
\hline \\ [-2.0ex]
 Evaluation			&19/11/2012		&22/11/2012		&30			&		\\
 Presenation			&19/11/2012		&21/11/2012		&30			&		\\
\hline \\ [-2.0ex]
 \bf{Total}			&				&				&\bf{1400}	&		\\
\hline
\end{tabular}
\label{table:wbs}
\end{table}

\section{Quality Assurance}
\subsection{Communication rules}
Communication with the customer is usually done by email. The customer will contact the assigned customer contact if anything else is not specified. The assigned customer contact is responsible for relaying any information received from the customer to the other group members as soon as possible. The customer contact is also responsible for sending any information from the group to the customer. Any communication from the customer demanding a reply, will be replied to within 8 hours. If any communication is sent from the group to the customer demanding a reply, this reply should be received by the group within 24 hours.

The same rules apply to the communication with the advisor.

\subsection{Internal Routines}
Each group member is responsible for logging all their work hours in the timekeeping system within 22:00 the same day. All group members are also responsible for checking the Skype group conversation within 22:00 Monday to Thursday, to see if any important information is posted there. If important information has to be delivered after this time, it will be sent by mail. All group members are required to check Redmine each day to check if any new task are assigned to that person.

\subsection{Deliverables}
All deliverables, including source code, documents and meeting notes, will be approved by at least two of the group members before it is sent for approval by either the advisor or the customer. All major phase documents must be approved by all the group members before it is relayed to the advisor for approval.

\subsection{Meetings}

\subsubsection{Advisor meetings}
Advisor meetings will occur every Thursday 09:15, unless anything else is specified, so the advisor gets updated on the latest work and progress of the project. The day before the meeting the group sends the advisor an agenda for the meeting and a status report of what is done since the last meeting. This will be sent to the advisor within 14:00 the day before the meeting. If this deadline is not sustained, any needed documents will be printed out and brought to the meeting. This lets the advisor enter the meeting prepared, and improve the efficiency of the meeting. Under the meeting issues regarding the work done, plans for next week and project management will be discussed. If there is a sprint-end-week the demo for this sprint will also be shown. Advisor notes will be written during the meetings, compiled and sent back to the advisor within 12:00 the following day.

\subsubsection{Customer meetings}
Customer meetings will occurs on demand, but usually weekly, and Thursdays 19:15 if there is a sprint-end-week, unless anything else is specified. Meetings will be scheduled at least 48 hours before the meeting starts, and confirmation from the customer should be received at least 24 hours before. An agenda for the meeting and a weekly progress report will be sent to the customer to keep him in the loop of the progress, and get prepared before the meetings. This documentation, including other documents needed for the meeting, will be sent by mail within 24 hours before the meeting. Since the customer is busy during work hours, the meetings are kept after 18:00 on weekdays, and after 12:00 in weekends. They will be kept over Skype, since the customer is located in Oslo. The meetings will be used to get the customer in the loop, clarify uncertainties and redirecting, if the direction of the project taken is not what the customer had in mind. The customer should also be included in the sprint planning meetings to help with use case prioritization. Customer notes will be written during the meetings, compiled and sent back to the customer within 12:00 the following day.

\subsubsection{Internal meetings}
Internal meetings will occur every Monday 12:15, unless anything else is specified. At this time all the group members are open for meetings. During these meetings the plans for the week will be discussed, workload divided and what was done last week. Notes will be taken and documented for later review. The group meet daily on skype to keep everyone up to date on the project progress and what will be done.

\subsubsection{Sprint planning meetings}
Sprint planning meetings will be held in the start of every sprint. Which use cases to handle and complete will be discussed here. The tasks will also be discussed with the customer so the customer can help prioritise the use cases and tell us which tasks they want us to complete first. The selected user stories along with their Work Breakdown Structure will be sent to the customer for approval.

\subsubsection{Sprint demonstration meetings}
Sprint demonstration meetings will be held Thursdays 19:15 in the end of the each sprint, unless anything else is specified. The group and the customer will be attending these meetings. An agenda will be sent to the customer, together with the weekly report and a link to the system, for the customer to test out. The demo will show the system and its new functionality. This lets the customer see the progress, and be able to come with inputs towards how their minds might differ from what has been produced, so the group can get on the right track if that is needed. This lets us make sure that we do not stray too far from the intended system. Since this is a meeting held with the customer the rules from the customer meetings will be held, so customer notes will be written during the meetings, compiled and sent back to the customer.

\subsection{Version Control}
GitHub was selected as the tool for version control in this project. All the relevant work that is produced will be pushed to the repository using git, including all the documents for the report, source code, images, diagrams, and so on. The group members will commit and push their changes on a regular basis, ensuring that the repository is always up to date and available.

\subsection{Document Templates}
The group has created templates for the following documents:

\begin{itemize}
\item Weekly status report
\item Meeting agenda
\item Meeting notes
\end{itemize}

These templates are listed in the appendices.

\begin{table}
\begin{tabularx}{\textwidth}{ | l | X | l | l | }
  \hline
  \textbf{\#} & \textbf{Risk} & \textbf{Probability} & \textbf{Impact} \\ \hline
  1 & Not getting a fifth party member & M & Significant \\ \hline
  2 & Obtrusive health/family/personal issues for team members & L & Significant \\ \hline
  3 & Low morale in team & M & Significant \\ \hline
  4 & Interfering workload from other activities & H & Minor \\ \hline
  5 & Miscommunication with customer & M & Critical \\ \hline
  6 & Changes in customer requirements & M & Significant \\ \hline
  7 & Errors in project plan & M & Significant \\ \hline
  8 & Failure of communication in team & M & Critical \\ \hline
  9 & Failure of time management & H & Critical \\ \hline
 10 & Errors in workload estimation and distribution & H & Critical \\ \hline
 11 & Failure of online storage systems and services & L & Significant \\ \hline
 12 & Failure of personal computers & M & Significant \\ \hline
 13 & Infeasibility of project as a whole & L & Critical \\ \hline
 14 & Inability to find potential users and test subjects & M & Significant \\ \hline
\end{tabularx}
\caption{Risks overview}
\end{table}


\begin{table}
\begin{tabularx}{\textwidth}{ | l | X | }
\hline
\textbf{Risk \#} & 01 \\ \hline
\textbf{Activity} & All \\ \hline
\textbf{Risk Factor} & Not getting a fifth party member \\ \hline
\textbf{Impact} & Significant \\ \hline
\textbf{Consequence} & Increased workload for all remaining party members on all activities \\ \hline
\textbf{Probability} & Medium \\ \hline
\textbf{Countermeasures} & \begin{itemize}
  \item  Contact advisor about the dropped party member, try to get assigned a new member.
  \item Take the missing person into account in planning phase.
\end{itemize}  \\ \hline
\textbf{Deadline} & Intro/Planning (Ultimately in the hands of course staff) \\ \hline
\textbf{Responsible} & Project leader \\ \hline
\end{tabularx}
\caption{Risk 01}
\end{table}

\medskip

\begin{table}
\begin{tabularx}{\textwidth}{ | l | X | }
\hline
\textbf{Risk \#} & 02 \\ \hline
\textbf{Activity} & All \\ \hline
\textbf{Risk Factor} & Obtrusive health/family/personal issues for team members \\ \hline
\textbf{Impact} & Significant \\ \hline
\textbf{Consequence} & Increased workload for all remaining party members on all activities \\ \hline
\textbf{Probability} & Low  \\ \hline
\textbf{Countermeasures} & \begin{itemize}
  \item Implement buffers in project plan.
  \item Team members should make their work resumable by another member.
\end{itemize}  \\ \hline
\textbf{Deadline} &  None \\ \hline
\textbf{Responsible} & Project leader \\ \hline
\end{tabularx}
\caption{Risk 02}
\end{table}

\medskip

\begin{table}
\begin{tabularx}{\textwidth}{ | l | X | }
\hline
\textbf{Risk \#} & 03 \\ \hline
\textbf{Activity} & All \\ \hline
\textbf{Risk Factor} & Low morale in team \\ \hline
\textbf{Impact} & Significant \\ \hline
\textbf{Consequence} & Decreased overall project quality \\ \hline
\textbf{Probability} & Medium  \\ \hline
\textbf{Countermeasures} & \begin{itemize}
  \item Frequent contact between team members
  \item Avoid team members overworking
  \item Focus on general team dynamics advice from advisor
\end{itemize}  \\ \hline
\textbf{Deadline} &  None \\ \hline
\textbf{Responsible} & Project leader \\ \hline
\end{tabularx}
\caption{Risk 03}
\end{table}

\medskip

\begin{table}
\begin{tabularx}{\textwidth}{ | l | X | }
\hline
\textbf{Risk \#} & 04 \\ \hline
\textbf{Activity} & All \\ \hline
\textbf{Risk Factor} & Interfering workload from other activities \\ \hline
\textbf{Impact} & Low \\ \hline
\textbf{Consequence} & Work on the project is shifted in time, space and responsibility \\ \hline
\textbf{Probability} & Very High \\ \hline
\textbf{Countermeasures} & \begin{itemize}
  \item Plan ahead with respect to existing schedules
  \item Inform the group of other activities
\end{itemize}  \\ \hline
\textbf{Deadline} &  None \\ \hline
\textbf{Responsible} & Project leader \\ \hline
\end{tabularx}
\caption{Risk 04}
\end{table}

\medskip

\begin{table}
\begin{tabularx}{\textwidth}{ | l | X | }
\hline
\textbf{Risk \#} & 05 \\ \hline
\textbf{Activity} & All \\ \hline
\textbf{Risk Factor} & Miscommunication with customer \\ \hline
\textbf{Impact} & Critical \\ \hline

\textbf{Consequence} & -The project is not developed as the customer wants it
-Work has to be done over \\ \hline
\textbf{Probability} & Very High \\ \hline
\textbf{Countermeasures} & \begin{itemize}
  \item Weekly customer meetings
  \item Share as much information as possible with customer at all stages
\end{itemize}  \\ \hline
\textbf{Deadline} &  None \\ \hline
\textbf{Responsible} & Customer Contact \\ \hline
\end{tabularx}
\caption{Risk 05}
\end{table}

\medskip

\begin{table}
\begin{tabularx}{\textwidth}{ | l | X | }
\hline
\textbf{Risk \#} & 06 \\ \hline
\textbf{Activity} & Planning, Requirements, Implementation \\ \hline
\textbf{Risk Factor} & Changes in customer requirements \\ \hline
\textbf{Impact} & Significant \\ \hline
\textbf{Consequence} & Work has to be done over  \\ \hline
\textbf{Probability} & Medium \\ \hline
\textbf{Countermeasures} & \begin{itemize}
  \item Design the prototype with possible modifications in mind.
  \item Try to get information on possible changes from the customer.
\end{itemize}  \\ \hline
\textbf{Deadline} &  None \\ \hline
\textbf{Responsible} & Customer Contact \\ \hline
\end{tabularx}
\caption{Risk 06}
\end{table}

\medskip

\begin{table}
\begin{tabularx}{\textwidth}{ | l | X | }
\hline
\textbf{Risk \#} & 07 \\ \hline
\textbf{Activity} & Implementation \\ \hline
\textbf{Risk Factor} & Errors in project plan \\ \hline
\textbf{Impact} & Significant \\ \hline
\textbf{Consequence} & Work on the plan and implementation have to be redone  \\ \hline
\textbf{Probability} & Medium \\ \hline
\textbf{Countermeasures} & \begin{itemize}
  \item Review the project plan frequently for consistency
  \item Share plan with customer
\end{itemize}  \\ \hline
\textbf{Deadline} &  Planning \\ \hline
\textbf{Responsible} & Project Leader \\ \hline
\end{tabularx}
\caption{Risk 07}
\end{table}

\medskip

\begin{table}
\begin{tabularx}{\textwidth}{ | l | X | }
\hline
\textbf{Risk \#} & 08 \\ \hline
\textbf{Activity} & All \\ \hline
\textbf{Risk Factor} & Failure of communication in team \\ \hline
\textbf{Impact} & Critical \\ \hline
\textbf{Consequence} & Failure of unification of the work, uneven workloads, decreased project quality  \\ \hline
\textbf{Probability} & Medium \\ \hline
\textbf{Countermeasures} & \begin{itemize}
  \item Frequent internal meetings
  \item Sharing of work internally
\end{itemize}  \\ \hline
\textbf{Deadline} &  None \\ \hline
\textbf{Responsible} & Project Leader \\ \hline
\end{tabularx}
\caption{Risk 08}
\end{table}

\medskip

\begin{table}
\begin{tabularx}{\textwidth}{ | l | X | }
\hline
\textbf{Risk \#} & 09 \\ \hline
\textbf{Activity} & All \\ \hline
\textbf{Risk Factor} & Failure of time management \\ \hline
\textbf{Impact} & Critical \\ \hline
\textbf{Consequence} & Parts of project are rushed or not finished in time \\ \hline
\textbf{Probability} & High \\ \hline
\textbf{Countermeasures} & \begin{itemize}
  \item Put in as much work as possible as early as possible
  \item Implement buffers in project plan
\end{itemize}  \\ \hline
\textbf{Deadline} &  None \\ \hline
\textbf{Responsible} & Project Leader \\ \hline
\end{tabularx}
\caption{Risk 09}
\end{table}

\medskip

\begin{table}
\begin{tabularx}{\textwidth}{ | l | X | }
\hline
\textbf{Risk \#} & 10 \\ \hline
\textbf{Activity} & All \\ \hline
\textbf{Risk Factor} & Errors in workload estimation and distribution \\ \hline
\textbf{Impact} & Critical \\ \hline
\textbf{Consequence} & Uneven workloads, rushed or unfinished parts of project \\ \hline
\textbf{Probability} & High \\ \hline
\textbf{Countermeasures} & \begin{itemize}
  \item Implement buffers in project plan
  \item Avoid relying too much on rigid plans
  \item Allow for redistribution of work when necessary
\end{itemize}  \\ \hline
\textbf{Deadline} &  Planning \\ \hline
\textbf{Responsible} & Project Leader \\ \hline
\end{tabularx}
\caption{Risk 10}
\end{table}

\medskip

\begin{table}
\begin{tabularx}{\textwidth}{ | l | X | }
\hline
\textbf{Risk \#} & 11 \\ \hline
\textbf{Activity} & All \\ \hline
\textbf{Risk Factor} & Failure of online storage systems and services \\ \hline
\textbf{Impact} & Critical \\ \hline
\textbf{Consequence} & Work is lost and has to be recreated \\ \hline
\textbf{Probability} & Low \\ \hline
\textbf{Countermeasures} & \begin{itemize}
  \item Local backups of data
  \item Know of alternative systems in case of failure
\end{itemize}  \\ \hline
\textbf{Deadline} &  None \\ \hline
\textbf{Responsible} & Project Leader \\ \hline
\end{tabularx}
\caption{Risk 11}
\end{table}

\medskip

\begin{table}
\begin{tabularx}{\textwidth}{ | l | X | }
\hline
\textbf{Risk \#} & 12 \\ \hline
\textbf{Activity} & All \\ \hline
\textbf{Risk Factor} & Failure of personal computers \\ \hline
\textbf{Impact} & Significant \\ \hline
\textbf{Consequence} & Work may be lost, decreased productivity of team member \\ \hline
\textbf{Probability} & Medium \\ \hline
\textbf{Countermeasures} & \begin{itemize}
  \item Use primarily online storage systems and keep online backups of everything else
  \item Use university computers if necessary
\end{itemize}  \\ \hline
\textbf{Deadline} &  None \\ \hline
\textbf{Responsible} & Individual \\ \hline
\end{tabularx}
\caption{Risk 12}
\end{table}

\medskip

\begin{table}
\begin{tabularx}{\textwidth}{ | l | X | }
\hline
\textbf{Risk \#} & 13 \\ \hline
\textbf{Activity} & All \\ \hline
\textbf{Risk Factor} & Infeasibility of project as a whole \\ \hline
\textbf{Impact} & Critical \\ \hline
\textbf{Consequence} & The concept is not a solution to the problem and the prototype is destined to be a failure \\ \hline
\textbf{Probability} & Very Low \\ \hline
\textbf{Countermeasures} & \begin{itemize}
  \item Extensive preliminary study to uncover this as early as possible
\end{itemize}  \\ \hline
\textbf{Deadline} & Feasibility study \\ \hline
\textbf{Responsible} & Project Leader \\ \hline
\end{tabularx}
\caption{Risk 13}
\end{table}

\medskip

\begin{table}
\begin{tabularx}{\textwidth}{ | l | X | }
\hline
\textbf{Risk \#} & 14 \\ \hline
\textbf{Activity} & Planning \\ \hline
\textbf{Risk Factor} & Inability to find potential users and test subjects \\ \hline
\textbf{Impact} & Significant \\ \hline
\textbf{Consequence} & Requirements engineering and prototype testing will be sub-standard unable to provide adequate answers \\ \hline
\textbf{Probability} & Medium \\ \hline
\textbf{Countermeasures} & \begin{itemize}
  \item Try to get information on potential users from customer
  \item Begin contacting potential users and testers early
\end{itemize}  \\ \hline
\textbf{Deadline} & Testing \\ \hline
\textbf{Responsible} & Project Leader \\ \hline
\end{tabularx}
\caption{Risk 14}
\end{table}



\section{Planned Effort}
The course staff recommends us to work 25 person-hours per week and student. This project is estimated for 14 weeks. Since we at the moment have 4 group members in our group, the available effort will be $14*25*4=1400$ person hours including own reading, meetings, lectures, and seminars. The customer requested 5-7 students to handle this project, it is regrettably not likely that we will be supplied by one extra group member, so we must expect some more work hours divided on the four of us.

\section{Lectures}
Lectures held by the course TDT4290 Customer Driven Project mainly aim to educate the students in the in the art of project management and the different tools used to handle this task in a better way. The lectures can give good insight into the workflow and how to avoid different issues, and should therefore be attended when the lecture involves tools used by the group or tools the group should consider using. After the lectures the group will reflect upon the presented material and if it is usable, or add value to our project, it will be integraded into the project.

\section{Issues}
\subsection{Issues regarding  the system}
The customer has a huge workforce at their disposal, and it can be introduced to us through the customer. If there would appear a problem regarding the system we can’t handle ourselves, we can contact the customer and they can forward the issue to other sections of the corporation for analysis and problem solving. The communication regarding issues of this kind will usually be done per email. 

\subsection{Issues regarding the workflow}
The advisor can help us with issues regarding project management and workflow issues. If the group were to be stuck at some point, the advisor can jank the group out of the ditch and set it back on track. 

\section{Tool Selection}
After detailed study of some tools, we decided to use tools and frameworks listed in table \ref{table-tools}. In communication and collaboration tools section, the main reason for choosing these tools were last experiences with these - most of us already used these tools in work or in other projects, so we do not have to learn to use something new. 

LaTeX software was used to write this report, although most of the team members did not use LaTeX before We decided to use it, because it has a lot of advantages. It is easy to colaborate on source code and use git to track the changes, the result pdf document looks better, than the one produced in other tools, there is automatic referencing of tables, figures and citations. Also the table of contents, tables and figures is automatically generated. Also, using LaTeX was recommended by the customer.

For backend of the system, we wil be using MongoDB in combination with Node.js. Both are extremely effective, scalable and easy to learn. Also they share the same runtime, so the deployment process on the server is simplified. MongoDB and Node.js are newand up-to-date technologies, that are getting more and more used in web applications. NoSQL schemeless MongoDB allows to store any data frontend requires and also the data migration is easy. Node.js can communicate with MongoDB using JavaScrip libraries and offer the data through standard REST interface. Client can retrieve the data from server via HTTP protocol using Ajax with JavaScript and present it to the user. The user interface is written using HTML, CSS and JavaScript.

\begin{table}
\centering
\begin{tabular}{ l l}
\textbf{Part of project}  & \textbf{Technology} \\
\hline
Communication & Skype, Google Groups, Google Calendar \\
Document colaboration & Google Docs, Google Drive \\
Code colaboration & Git + Github \\
Time tracking and mamangement & Redmine \\
Report & Latex \\
Backend & Node.js \\
Database & MongoDB \\
Communication & PubNub \\
Client & JavasScript + JQuery, HTML, CSS \\
\hline
\end{tabular}
\caption{Tools used in project}
\label{table-tools}
\end{table}

\section{Choice of Domain}
Our project's problem is a rather general one, not intrinsically tied to any specific domain. However, its existence depends on an actual area of application. For this reason, it has been necessary for us to discover a domain for which to implement our prototype. Our criteria for selecting such a domain were as follows:
\begin{itemize}
\item It should be simple enough for test subjects to understand and feel familliar with.
\item It should be complicated enough to demonstrate the problem.
\item The console has to be useful in the domain, in such a way that it eases the workflow or opens new possibilities
\item Object oriented design should be applicable
\item There should be potential for the existence of power users capable of using the console
\end{itemize}
Various domains were considered, including but not limited to banking, warehouses, project management, travel agencies, education, social networking, music management and health care. Ultimately, based on the criteria listed, we decided in consensus to implement our solution for a library system.

\begin{comment}
Domain ideas:
project management tool (ehm. redmine)
warehouse
airport / travel agency
cash registers (not interesting)
bank
school information system (boring)
time management system (aka calendar)
facebook console (unreal)
mail interface (gmail is too good, not useful)
photo gallery (limited)
music library (limited)
physician system (treatments, prescriptions, add person)
tax form (boring)
Building management (boring)
Library

(ehm. redmine) 
warehouse 1
1
bank 3 3
(aka calendar) 2
(treatments, prescriptions, add person) 1
Library 2 2 3

Ivo	Martin	Oddvar	Oystein	Total
project management tool	3		2		5
warehouse			1		1
airport / travel agency		1			1
bank		3		3	6
time management system	2				2
physician system	1			1	2
Library		2	3	2	7




Top3:
library
authors, books(location, state), book management, employees, user management, order new books, 

bank
accounts, transfer money, deposits, loans, customer management, employees

project management
projects, issues, work done on issues, workers, sprints, reports, 

interesting	simple to understand	complex enough	ease workflow	existence of power user	total
library	2	3	1	2	Y	8
bank	2	3	2	2	Y	9
project management	1	2	3	2	Y	8



project management
+ usefulness
+ field of study
- boring

library
+ libraries in campus, employees
- simple

bank
- unknown ground


library	project management	bank
martin	3	2	1
oddvar	3	2	1
ivo	2	3	1
oystein	3	1	2
total	11	8	5



Criteria:
Interesting and simple enough for test subjects to understand. something they meet in everyday life
Complicated enough to demonstrate the problem.
Console has to be useful in the domain, ease the workflow
Object oriented design
Existence of power user, that will use the console

Customer ideas:
car models
finn.no
health sector
human resources (people, how they are related, symptoms, how much money and what they own - advanced domain graph). 
contact lists (peder)


write these things to the report


Future:
batch operations
undo the commands
save command for object attribute editing
what about object operations?
transaction support
\end{comment}

\section{Project name}
Project name is important project identificator. It should summarize main project goal or functionality. In real project, is often a trademark, or reflects the name of the company.

In this section, we will describe the process of choosing a name for the project. First of all, we made a list of words, that we can use in the project name. These words describe project funcionality or goal.

Words that can be used in project name: master, console, web, text, keyboard.

We used the list of words and brainstorming session on a meeting for compiling a list of project name candidates:

Candidate project names: console 2.0, wonsole, wensole, websole, werminal, interCLI.

After discussion we chose the name \emph{Wonsole}. Project name can be sometimes little confusing, so we added the subtitle: \emph{The new web console for power users.}
