\chapter{Sprint 3}
\section{Planning}
\subsection{Duration}
\subsection{Sprint Goal}
\subsection{User stories}
\section{Architecture}
\subsection{4+1 view model}
\subsubsection{Logical View}
\subsubsection{Development View}
\subsubsection{Process View}
\subsubsection{Physical View}
\section{Implementation}
\section{Testing}
\subsection{Test Results}
\subsection{Test Evaluation}
\section{Customer Feedback}

\subsection{Before}
At the end of sprint 2 the customer representative informed us that he would have a meeting with others in the company, to try to identify what our product was missing, a X- factor that would appeal to the users and promote the console.

At the Scrum meeting for the third sprint he announced that they had indeed found this X- factor. He wanted us to return to the roots of the project, namely adding scripting to a web- page. We should focus on adding functionality which is impossible or difficult and time consuming to do in a regular GUI.

He wanted us to add functionality that would show of the advantages of the technologies we are using, and how it would not be possible to do this with more traditional technologies. He listed some general use cases he wanted us to implement, and suggested that we looked into another solution for the backend. This was because he felt that this solution was better suited for the use cases he now presented us. He was aware of the time constraints of the project and did not expect us to manage to implement all the functionality he mentioned. But it was important that we could document that it would be possible to implement it with the technologies we are using in this project.

\subsection{After}

\section{Evaluation}
\subsection{Review}
\subsection{Positive}
\subsection{Negative}
\subsection{Planned Responses}
