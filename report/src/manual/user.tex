\section{Introduction}
This document describes how to use the Wonsole application. It contains a user
tutorial as well as a command refence.

\section{Basics}
The user interface for wonsole is minimal and simple, so that the functionality gets
presented in a clear way. If you see the main screen in the
picture \ref{wonsole2-00}, on the left hand side, there is a GUI - Graphical
User Interface - this is where data get presented as graphics. On the right hand side, there is a console window. In this location,
the user can input commands and send them to the system by
pressing enter.

By default, the list of available databases is displayed in the GUI and
a command prompt in the console.


\begin{figure}
\centering
\includegraphics[width=\textwidth]{../../manual/screenshot/wonsole2/wonsole2-00.png}
\caption{Main screen}
\label{wonsole2-00}
\end{figure}

\section{Tutorial}

\subsection{Database: db}
The most basic command in Wonsole is the command to open, or switch database. It
has this format:
\begin{verbatim}
db DATABASE
\end{verbatim}
Where DATABASE is the name of the database to open. Databases are listed in the GUI
and they can be opened by clicking on the name as well.

After issuing the command, the database contents are displayed in the GUI, see
picture \ref{wonsole2-02}.
Wonsole uses a document oriented database, so it is list of documents indexed by numbers.
From each document the title attribute is displayed as preview.


\begin{figure}
\centering
\includegraphics[width=\textwidth]{../../manual/screenshot/wonsole2/wonsole2-02.png}
\caption{Database open}
\label{wonsole2-02}
\end{figure}

The same command can be used to switch the database anywhere in the workflow as in
picture \ref{wonsole2-04}.

\textit{Note: All uncommited changes are lost when switching the database.}

\begin{figure}
\centering
\includegraphics[width=\textwidth]{../../manual/screenshot/wonsole2/wonsole2-04.png}
\caption{Database switch}
\label{wonsole2-04}
\end{figure}

\subsection{Presentation: print and view}
The command print is available to display data in plaintext instead of using the GUI.
Its syntax is very simple - it has one attribute: the variable, expression or data to print.
\begin{verbatim}
print DATA
\end{verbatim}
The command loads the data, converts it into text and prints in the console as ssen
in picture \ref{wonsole2-10}.

\begin{figure}
\centering
\includegraphics[width=\textwidth]{../../manual/screenshot/wonsole2/wonsole2-10.png}
\caption{Print}
\label{wonsole2-10}
\end{figure}

Opposite of the print command there is the command view. It displays a complex variable in
the GUI.
\begin{verbatim}
view VARIABLE
\end{verbatim}

\subsection{JavaScript integration}
Wonsole includes the complete JavaScript language to the command set. If user issues
a command, that is not recognized as a Wonsole command, this input gets
evaluated as JavaScript. You can for example issue a helper command, that does
some required functionality not supported by Wonsole and then continue using
default Wonsole commands. The JavaScript language is 100% integrated into Wonsole.

\begin{verbatim}
var new = "New variable" //Javascript
print variable
"New variable"
\end{verbatim}

This is an example of creating a new JavaScript variable and using it in a simple
Wonsole command, demonstration in picture \ref{wonsole2-15}.

\textit{Note: This is advanced functionality. The JavaScript language is very
complex and it is the user's full responsibility to use it in a safe, suitable
way.}
 

\begin{figure}
\centering
\includegraphics[width=\textwidth]{../../manual/screenshot/wonsole2/wonsole2-15.png}
\caption{Javascript integration}
\label{wonsole2-15}
\end{figure}

\subsection{Display: Quiet}
When Wonsole displays a list of documents saved in the database (see command
db) by default it uses quiet mode. In quiet mode, it displays only object index and
value of the title attribute as a object identification as you can see in
picture \ref{wonsole2-18}.

\begin{figure}
\centering
\includegraphics[width=\textwidth]{../../manual/screenshot/wonsole2/wonsole2-18.png}
\caption{Quiet mode on}
\label{wonsole2-18}
\end{figure}

\begin{verbatim}
quiet
\end{verbatim}

You can toggle preview mode by using command quiet. When toggled off, all
attributes of the object are displayed - but only the simple (primitive) data types. If the
attribute is complex, the Object keyword is displayed instead, see picture
\ref{wonsole2-20}.

\begin{figure}
\centering
\includegraphics[width=\textwidth]{../../manual/screenshot/wonsole2/wonsole2-20.png}
\caption{Quiet mode off}
\label{wonsole2-20}
\end{figure}


\subsection{Documents: docs and doc}
After opening the database, the list of documents is loaded. Is is stored in the
\verb|docs| variable for you to work with. This variable is an array, so you can
index it and get a single document object as seen in picture \ref{wonsole2-26}.

\begin{verbatim}
db books
docs
[...]
\end{verbatim}
 

\begin{figure}
\centering
\includegraphics[width=\textwidth]{../../manual/screenshot/wonsole2/wonsole2-26.png}
\caption{List of documents}
\label{wonsole2-26}
\end{figure}

Documents displayed in the document list mode are indexed using numbers. You can
use these numbers together with \verb|doc| command to open a single document.
This command can be also used to switch between documents, you just have to
issue it with different index. The opened document is stored inside a \verb|doc|
variable as demonstrated in picture \ref{wonsole2-32}.

\begin{figure}
\centering
\includegraphics[width=\textwidth]{../../manual/screenshot/wonsole2/wonsole2-32.png}
\caption{Doc object}
\label{wonsole2-32}
\end{figure}

To change the document, issue an assignment statement like you would using plain JavaScript. This way you can modify
the object freely, you can also add a new attributes if you wish. An example is given in the picture
\ref{wonsole2-38}, where the document is given a stock of 4, and is assigned an attribute "available" which is set to true.


\begin{figure}
\centering
\includegraphics[width=\textwidth]{../../manual/screenshot/wonsole2/wonsole2-38.png}
\caption{Modification of doc}
\label{wonsole2-38}
\end{figure}



\subsection{Transactions: Commit and Rollback}
All changes you perform on objects are local, so at any point during the
modification process, transaction, you have the option to save all the changes to the
database using the command \verb|commit| or to throw away all local changes and load
back the values from database with command \verb|rollback|. This functionality
can be used as an undo command. It is up to you how often you commit the
changes.


\subsection{Objects: Add and Remove}
To add a new object, use an \verb|add| command. It has one parameter, the object to
add. If you are not sure, just create an empty JavaScript object using the empty curly braces. Otherwise you can
directly define specific object attributes. Examples are given in picture \ref{wonsole2-48}. The
new object will immediatelly appear in document list, so you can open
it and continue working with it, for example add a new attribute as in picture
\ref{wonsole2-48}.


\begin{figure}
\centering
\includegraphics[width=\textwidth]{../../manual/screenshot/wonsole2/wonsole2-48.png}
\caption{Add}
\label{wonsole2-48}
\end{figure}

\begin{figure}
\centering
\includegraphics[width=\textwidth]{../../manual/screenshot/wonsole2/wonsole2-55.png}
\caption{Add and modify}
\label{wonsole2-55}
\end{figure}
